\documentclass[10pt,a4paper]{article}
\usepackage{fontspec}
\usepackage{hyperref}
\usepackage{graphicx}
\defaultfontfeatures{Ligatures={TeX}}
\setmainfont{CMU Serif}
\setsansfont{CMU Sans Serif}
\setmonofont{CMU Typewriter Text}

\usepackage[top=1cm, right=1cm, bottom=1cm, left=1cm]{geometry}
\thispagestyle{empty}

\title{Останови войну!}


\begin{document}
	
	\begin{figure}[t]
		\centering
		\includegraphics[width=80pt]{./pacific.png}
		\caption{Peace sing}
		\label{fig:Pacific_sign}
	\end{figure}
	
	
	
	\begin{center}
		{\LARGE Останови войну!}
	\end{center}
	
	\section{Россия ведет агрессивную войну против Украины.}
	
	\section{На свободе ты человек со всеми правами человека.
	Принимая участие в боевых действиях на стороне России ты
	несешь смерть и разрушение.
	С оружием в руках или внутри боевой машины ты источник
	смертельной опасности и лишь цель для высокоточного оружия,
	который будет поражен.}
	
	\section{Россия является заитересованной стороной,
	все ее действия направлены только на достижение военных целей,
	только мировое сообщество способно остановить конфликт.
	Победа Украины (возврат ее признанных ООН территорий
	и ее деокупация) - победа свободной России.}
	
	\section{Во второй чеченской войне погибло от 50000
	до 80000 человек, в первой - от 90000 до 100000 человек}
	
	\href{https://en.wikipedia.org/wiki/First\_Chechen\_War}
	{https://en.wikipedia.org/wiki/First\_Chechen\_War}
	
	\section{До вмешательства НАТО (сегодня это альянс 30 стран)
	в военном конфликте в Югославии погибло до 140000 человек}
	
	\href{https://en.wikipedia.org/wiki/Yugoslav\_Wars}
	{https://en.wikipedia.org/wiki/Yugoslav\_Wars},
	\href{https://en.wikipedia.org/wiki/NATO\_bombing\_of\_Yugoslavia}
	{https://en.wikipedia.org/wiki/NATO\_bombing\_of\_Yugoslavia}

	Операция НАТО, которая велась высокоточным оружием
	и привела к остановке войны и суду над Милошевичем сопровождалась
	гибелью около 500 человек.
			
	\section{В военном конфликте на Украине погибло до 331000 человек}
	
	И Российская Федерация продолжает вести военные действия.
	Останься живым!
	
	\href{https://en.wikipedia.org/wiki/Russo-Ukrainian\_War}
	{https://en.wikipedia.org/wiki/Russo-Ukrainian\_War}
	
	Помни! Любой должен понимать, что перед ним не послушный винтик
	системы, а сознательный гражданин.
	Отказаться от несения военной службы с заменой её на
	альтернативную гражданскую можно на основании части 3 статьи 59
	Конституции РФ.
	Отказаться от братоубийства можно (и нужно) на основании ст. 18
	Международного пакта о гражданских и политических правах.
	Не воспринимай экстремистскую литературу (Лимонов, М. Калашников,
	пропоганду "левых" партий об угрозе НАТО и необходимости войны)
	как руководство к действию - это призывы к ведению агрессивной
	войны.
	
	\begin{figure}[b]
		\centering
		\includegraphics[width=70pt]{./pacific_girl.png}
		\caption{Peace!}
		\label{fig:pacific_girl}
	\end{figure}
	
\end{document}
