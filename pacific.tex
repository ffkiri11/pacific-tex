\documentclass[10pt,a4paper]{article}
\usepackage{fontspec}
\usepackage{hyperref}
\usepackage{graphicx}
\defaultfontfeatures{Ligatures={TeX}}
\setmainfont{CMU Serif}
\setsansfont{CMU Sans Serif}
\setmonofont{CMU Typewriter Text}

\usepackage[top=1cm, right=1cm, bottom=1cm, left=1cm]{geometry}


\title{Останови войну!}

\begin{document}

\begin{figure}[t]
  \centering
  \includegraphics[width=120pt]{./pacific.png}
  \caption{Peace sing}
  \label{fig:Pacific_sign}
\end{figure}



\begin{center}
{\LARGE Останови войну!}
\end{center}

\section{Война есть братоубийство!}

\section{Никогда не участвуй в военных конфликтах!}

\section{Военные конфликты на территории бывшего СССР - гражданская война.}

\section{Любые военные конфликты должны разрешаться мирным способом}

\section{Во второй чеченской войне погибло от 50000 до 80000 человек, в первой - от 90000 до 100000 человек}

\href{https://en.wikipedia.org/wiki/First\_Chechen\_War}
{https://en.wikipedia.org/wiki/First\_Chechen\_War}

\section{До вмешательства НАТО в военном конфликте в Югославии погибло до 140000 человек}

\href{https://en.wikipedia.org/wiki/Yugoslav\_Wars}
{https://en.wikipedia.org/wiki/Yugoslav\_Wars}.

Операция НАТО, которая велась высокоточным оружием и привела к остановке войны
и суду над Милошевичем сопровождалась гибелью около 500 человек.

\href{https://en.wikipedia.org/wiki/NATO\_bombing\_of\_Yugoslavia}
{https://en.wikipedia.org/wiki/NATO\_bombing\_of\_Yugoslavia}

\section{В военном конфликте на Украине погибло до 15000 человек}

И Российская Федерация продолжает вести военные действия. Останься живым!

\href{https://en.wikipedia.org/wiki/Russo-Ukrainian\_War}
{https://en.wikipedia.org/wiki/Russo-Ukrainian\_War}

Помни! Любой командир должен понимать, что перед ним не послушный винтик
системы, а сознательный гражданин.
Отказаться от несения военной службы с заменой её на
альтернативную гражданскую можно на основании части 3 статьи 59
Конституции РФ.
Отказаться от братоубийства можно (и нужно) на основании ст. 18
Международного пакта о гражданских и политических правах.

Не воспринимай экстремистскую литературу (Лимонов, М. Калашников, агитки
"левых" партий об угрозе НАТО и необходимости войны) как руководство к
действию - в ней программа по уничтожению тебя и твоих братьев.

\begin{figure}[b]
  \centering
  \includegraphics[width=100pt]{./pacific_girl.png}
  \caption{Peace!}
  \label{fig:pacific_girl}
\end{figure}

\end{document}

